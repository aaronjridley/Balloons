\documentclass[11pt]{article}

\textheight=9.0in
\textwidth=6.5in
\oddsidemargin=0.0cm
\evensidemargin=0.0cm
\topmargin=-0.5in

\begin{document}


\begin{center}
{\bf \Large Space 584 

Project Specifications and Requirements
}
\end{center}

\section{Overview}

The goal of our project is to take measurements with sensors of
atmospheric pressure, temperature, humidity and wind speed from the
ground to 100,000 ft. as well as take video and pictures of the
atmosphere and ground from altitude. Teams will integrated a number of
sensors with a microcontroller, log data obtained onboard for later
analysis, and design a tracking device that transmits positional
telemetry to the ground throughout the flight.  Teams must track and
retrieve the package.  This means that the system must contain a
flight termination unit that is either triggered from the ground or
can trigger on its own.  It is mandatory that all FAA regulations be
followed in the process.  Teams also must plan for a ride share on
their payload train with Engin 100 student payloads.  Teams must plan
to integrate up to three Engin 100 payloads into their final payload
train on launch day.

\section{System Requirements}

The payload components you are responsible for designing, building,
testing, and launching are: stand-alone Flight Termination Unit (FTU),
a radio tracker (e.g., Trackuino), and main science payload with
sensors.

\subsection{Tracker Requirements}

\begin{itemize}
  \item The tracker must broadcast its position approximately once a
    minute ($\pm 10s$) over 144.39 MHz in a format that can be decoded by
    the APRS network.
  \item At least two team members must be licensed to operate the
    tracker and ground station.
  \item At least two team members must be able to operate the ground
    station to receive data packets that upload properly to the APRS
    site.
  \item The battery should be sized to provide at least 2.5 hours of
    power, but should not be oversized due to weight constraints.
  \item Power must be controlled by a remove before flight pin or a
    switch so it can be initiated without opening the package at the
    launch site and remain on during the flight with no chance of it
    accidentally being turned off.
  \item Headers must be used for any components that are expensive, so
    they can be recovered and reused later.
  \item There should be connection points at the top and bottom of the
    payload to allow connection to the main train, with the weight of
    the packages below being supported. Conversely, the tracker could
    be connected with a single connection point at the top and hang
    off the main train. A connection point must be made available for
    a backup safety line to run through.
  \item The tracker must be less than 0.75 lbs. An extra 0.25 lbs is
    available if a camera is incorporated.
  \item The tracker must be able to operate after being subjected to
    extreme environments of acceleration and temperature.
  \item It is critical that GPS have lock during the flight and that
    transmission (and reception) occurs during the last minutes of the
    flight.
\end{itemize}

\subsection{Main Science Payload Requirements}

\begin{itemize}
  \item Record inside and outside temperature, pressure, humidity, and
    acceleration at least once every 10s during the flight.
  \item One additional sensor of your choosing beyond the basic
    requirements must be added.  You should justify why you want to
    add this sensor and describe how it will be used.  A budget of up
    to \$50/team is allowed for this sensor.
  \item Data should be recorded to an SD or Micro-SD card.
  \item It would be good to record GPS position on the main science
    payload, but it is not required.
  \item The battery should be sized to provide at least 2.5 hours of
     power, but should not be oversized due to weight constraints.
  \item Power must be controlled by a remove before flight pin or a
      switch so it can be initiated without opening the package at the
      launch site and remain on during the flight with no chance of it
      accidentally being turned off.
  \item Headers must be used for any components that are expensive,
      so they can be recovered and reused later.
  \item There should be connection points at the top and bottom of
      the payload to allow connection to the main train, with the
      weight of the packages below being supported. Conversely, the
      main science payload could be connected with a single connection
      point at the top and hang off the main train. A connection point
      must be made available for a backup safety line to run through.
  \item The main science payload must be less than 0.75 lbs. An
      extra 0.25 lbs is available if a camera is incorporated.
  \item The science payload must be able to operate during and after
    being subjected to extreme environments of acceleration and
    temperature.
\end{itemize}


\subsection{FTU Requirements}

\begin{itemize}
  \item The FTU will be launched in its own structure, mounted below
    the balloon, but above the parachute on the payload train.
  \item The working FTU must include a timer, a battery pack, and
    whatever circuitry you need to initiated a timed burn of the
    nichrome wire (or other technique) to cut down the package.
    \begin{itemize}
      \item The FTU could also be extremely sophisticated with a radio
        (e.g., X-Bee) connected to the main science payload that can
        trigger the flight termination. This could be triggered due to
        going outside of a given lat/lon box, ascent rate too slow,
        etc.
    \end{itemize}
  \item Power must be controlled by a remove before flight pin or a
    switch so it can be initiated without opening the package at the
    launch site and remain on during the flight with no chance of it
    accidentally being turned off.
  \item The battery should be sized to provide at least 2.0 hours of
    power for the system, and be able to provide power to operate the
    termination mechanism, but should not be too over-powered due to
    weight.
  \item Connection points to the payload train must be included in the
    design, so that {\bf the package does not need to be opened at the
      launch site.}
  \item The FTU needs to be robust against line stress, since the
    nichrome is easy to dislodge and/or break.
  \item FTU must weigh less than 0.5 lbs.
\end{itemize}

\subsection{System-Level Requirements}
    
The tracker and main science payload can be combined into one box or
can be built in 2 separate boxes.  If they are together, a single
battery pack and structure can be used. If a single payload is used,
the weight is restricted to 1.5 lbs with an extra 0.5 lbs for two
cameras.

\section{Lab Specifications}

This class will have mostly lab sessions. I expect students to show up
during class times to work with their fellow teammates. We will have
weekly team meetings to make sure that the teams are keeping up with
the plan.

We will have to order things online. You should work with Aaron to put
in orders.  He will consolidate the orders and try to minimize
shipping. If you absolutely HAVE to order something on your own
(highly discouraged!!!), the parts must be shipped to me/the class
here at the University or you will not be reimbursed. Pay attention to
shipping costs! We do not want to pay outrageous shipping costs for
last-minute orders!  Plan ahead!

The following list makes up the lab reports that you must complete in
order to get a grade and launch, and in those labs, there are a number
of requirements that need to be met:

\subsection{Lab 1:  Microcontroller, Sensors, and Data Logging}

{\bf Due by: Critical Design Review}

\noindent
{\bf Requirements:}

\begin{itemize}
   \item Microcontroller working with sensors (2xTemperature,
     pressure, humidity, acceleration). The sensors should all be
     polled at close to the same time to show that they are all
     working together.
    \item Lab should show wiring schematic. 
    \item Lab should include preliminary calibration of sensors. 
    \item Include calibration information for sensors, include plots
      of the calibration curves, data used to verify calibration.
    \item Microcontroller must write data to the data logger.
    \item Lab should show pictures of breadboard, screenshots of the
      readout of numbers reported from each sensor and plots of sensor
      numbers plotted in MATLAB, python, excel, IDL, etc.
    \item Lab should show plots of the data over time (2 hours minimum).
    \item Microcontroller code should be included. 
\end{itemize}

\subsection{Lab 2:  Microcontroller + GPS}

{\bf Due by: Critical Design Review}

\noindent
{\bf Requirements:}

\begin{itemize}
    \item Micro-controller working with the GPS, logging the data to
      the data logger. This may be challenging if only using the GPS
      on the tracker. The tracker may have to be modified to log the
      data, or a separate GPS may have to be flown.
    \item A program on a computer should be able to parse this data
      and plot it in MATLAB, python, excel, IDL, etc.  You can make
      great webpage that displays a google map.
    \item Include working code in the report.
    \item Sample data should be shown in the lab write up, along with
      plots of location.
    \item Lab should show wiring scheme.  
    \item HINT: Tiny GPS will help prevent memory issues.
    \item GPS data plotted on Google maps. Clearly label your start
      and end points (use pins).
   \item Lab should include a map that shows that you can walk or
     drive around campus and get a map in Google that shows your path.
\end{itemize}

\subsection{Lab 3:  Tracker}

{\bf Due by: Critical Design Review, finished tracker due by Flight
  Readiness Review}

\noindent
{\bf Requirements:}

\begin{itemize}
    \item DO NOT operate transmitter without radio antenna. (Strongly)
      Consider hot gluing antenna connections (GPS and transmitter) to
      make them more robust.
    \item Demonstrate that you have integrated your Trackuino (or
      whatever radio system your team uses), soldered it together,
      assembled in a structure with the GPS antenna point upward for
      integration in to the payload train.
    \item Tracker needs to reliably transmit GPS position to one of
      the ground stations continuously for 30 minutes, and broadcasts
      over APRS for 30 minutes (e.g. WXMU South Ann Arbor).
    \item Lab should clearly demonstrate this through the use of
      screenshots and/or videos.
    \item It would be very nice (not required) to integrate Trackuino
      (or other communication system) with sensor payload and transfer
      data over APRS.  Data from sensors and GPS coordinates
      transmitted over the radio to the ground station for a duration
      of 30 minutes (once per minute or two). Lab should include
      videos demonstrating functionality.
      \begin{itemize}
        \item This can be accomplished by adding a sensor stack onto
          the Trackuino, routing the sensors outputs to the Arduino,
          and altering the code to read, store, and transmitting the
          data.
        \item This saves a lot of weight on the payload, but may be
          difficult, due to altering other people's design.
      \end{itemize}
\end{itemize}

\subsection{Lab 4: Extra sensor/capability lab}

{\bf Due by: Critical Design Review}

\noindent
{\bf Requirements:}

\begin{itemize}
   \item  Write up motivation for your sensor/capability choice.
   \item Describe how it is integrated in the payload, calibrate it,
     test it over its operating range, log data from it, and include a
     wiring diagram for it.
    \item The sensor should be working on a breadboard.
    \item Follow same requirements for lab 1.
\end{itemize}

\subsection{Lab 5: Flight Termination Unit}

{\bf Due by: Critical Design Review}

\noindent
{\bf Requirements:}

\begin{itemize}
    \item Develop protoboard for the FTU.
    \item Demonstration should be a video that shows that something is
      cut down after a set time from across the room.
    \item This video should also confirm the timing is according to
      the setting in the code.
   \item Lab should show pictures of the completed circuit board with
     data showing that everything works.
    \item Include a hardware schematic and all code in your
      submission.
\end{itemize}

\subsection{Lab 6: PCB design}

{\bf Due by: Thursday before Spring Break}

\noindent
{\bf Requirements:}

\begin{itemize}
    \item Develop PCB board layout for the main payload, Trackuino,
      and the FTU.
    \item The payload PCB should incorporate the sensors, the data
      logger, the micro-controller, the X-Bee (if used), the payload
      and any supporting circuitry.
    \item Nothing that costs more than \$20 should be soldered to the
      PCB - headers should be mounted to the PCB and the components
      should be put into the headers.
    \item Lab should show pictures of the completed circuit board with
      data showing that everything works ok on it.
    \item All files needed to order PCBs must be reviewed by at least
      2 other teams and delivered before class on the Thursday before
      spring break.
    \begin{itemize}
        \item All independent boards should be combined into a single
          PCB in order to minimize cost.
    \end{itemize}
\end{itemize}

\subsection{Interface Control Document}

{\bf Due by: March 17th}

\noindent
{\bf Requirements:}

\begin{itemize}
    \item Work with 3-4 ENGR100 teams to design interfaces between
      their ride-share payloads and the payload train.
    \item These teams will be given requirements with options and they
      will write you a memo describing how they will be attaching to
      the train and how they will meet the requirements.
    \item You will verify that the design meets the requirement and
      modify if you need to, creating a formal interface control
      document.
    \item On the Wednesday/Thursday before launch, you should meet
      with the ENGR 100 teams to verify that they have complied with
      the ICD.
\end{itemize}

\subsection{Lab 7:  Tracking Test Results}

{\bf Due by: Flight Readiness Review}

\noindent
{\bf Requirements:}

\begin{itemize}
  \item At least two team members must be licensed Ham radio
    technicians.
  \item Demonstrate ability to set up ground station and attach it
      to your car. Verify that at least two members of your group
      understand the procedure to set up and work ground station.
    \item Demonstrate MBuRST provided Micro-track working with ground
      stations, broadcasting to APRS (using wifi or cell from the
      ground station laptop).
    \item Demonstrate your tracker communicating with the ground
      station by including a number of packets transmitted to it from
      screenshot on APRS.
    \item Lab write-up should include screenshots from aprs.fi that
      show that the Micro-track and tracker are broadcasting the
      correct call sign and that the ground station is working for at
      least 30 minutes while moving around in line of sight of
      receiver.
\end{itemize}

\subsection{Lab 8:  Camera endurance and cold test}

{\bf Due by: Flight Readiness Review}

\noindent
{\bf Requirements:}

\begin{itemize}
    \item Camera working and verified to work at $-40^\circ$
    \item Measure battery lifetime prior to cold test, during cold
      test, after cold test, report on performance.
    \item The lab should show that the camera can work for at least 60
      minutes at $-40^\circ$.
\end{itemize}

\subsection{Lab 9:  System Testing}

{\bf Due by: Flight Readiness Review}

\noindent
{\bf Requirements:}

\begin{itemize}
    \item An end-to-end test needs to be completed on the complete
      system, including cold testing ($-40^\circ$C for at least 2
      hour) and endurance testing (operating for at least 3 hours).
    \item 20 ft drop and kick test and fix anything that didn’t hold
      up.
    \item A car chase needs to be completed with results that show
      that the complete system worked flawlessly during the entire
      chase (this includes the ground station and all of the sensors).
    \item Video demonstration of the FTU test.
    \item You will not be allowed to launch until the tests are completed.
    \item Tests {\bf must be completed at least 24 hours before
      launch}, so they can be reviewed.
    \item A spreadsheet providing information on masses of packages
      must be provided.
    \item Launch day predictions with launch and landind sites must be
      provided.
\end{itemize}

\subsection{Lab 10: Final Report}

{\bf Due by: day/time of final exam}

\noindent
{\bf Final report should contain the following:}

\begin{itemize}
    \item A terse summary of the system design
    \item A summary of launch day activities
\begin{itemize}
        \item What worked and what did not?
\end{itemize}
    \item Map of balloon trajectory including google map and altitude profile
    \item Plots of the data recovered and analysis of results
    \item Video/images should be provided in a shared google drive
\end{itemize}



\end{document}
