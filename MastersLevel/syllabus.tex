\documentclass[11pt]{article}

\textheight=9.0in
\textwidth=6.5in
\oddsidemargin=0.0cm
\evensidemargin=0.0cm
\topmargin=-0.5in

\begin{document}


\begin{center}
{\bf \Large Space 584 

Syllabus - Winter 2022
}
\end{center}



\section{Locations and Times}

Tuesday/Thursday 12:30 - 2:30 PM \\
2424 CSRB (Lab: B516 CSRB)

\section{Instructor Information}

Prof. Aaron Ridley\\
ridley@umich.edu\\
contact over Slack as a primary means of communication

\section{Course Overview}

The goal of this class is to design, test, and build an actual
``satellite'' system.  The point of this is to take a project from
design through implementation, including thorough testing.  Students
form teams of 5-6 people with diverse backgrounds to design a weather
balloon sensor payload and position tracking system which will be
deployed at the end of the semester.  Students are required to manage
their team and track their progress towards course milestones
including launch of their balloon payload.  These types of balloon
ascend to approximately 100,000 ft (i.e., 20 miles), burst, and
descend back to the ground. The National Weather Service launches
approximately 200 of these balloons per day, although they are
radically different than the type that you will be building. They
track the balloons with radar (which we don’t have access to) and use
a simple radio for 1-way communication. They also weigh significantly
less than 1 kg.

\section{Course Objectives}

The goal of our project is to take measurements with sensors of
atmospheric pressure, temperature, humidity and wind speed from the
ground to 100,000 ft. as well as take video and pictures of the
atmosphere and ground from altitude. Teams will integrated a number of
sensors with a microcontroller, log data obtained onboard for later
analysis, and design a tracking device that transmits positional
telemetry to the ground throughout the flight.  Teams must track and
retrieve the package.  This means that the system must contain a
flight termination unit that is either triggered from the ground or
can trigger on its own.  It is mandatory that all FAA regulations be
followed in the process.  Teams also must plan for a ride share on
their payload train with Engin 100 student payloads.  Teams must plan
to integrate up to three Engin 100 payloads into their final payload
train on launch day.

\section{Course Structure}

This class will have primarily lab sessions. I expect students to show
up during class times to work with their fellow teammates. We will
have weekly team meetings to make sure that the teams are keeping up
with the plan.

We will have to order things online. You should work with Scott to put
in orders.  He will consolidate the orders and try to minimize
shipping. If you absolutely HAVE to order something on your own
(highly discouraged!!!), {\bf the parts must be shipped to me/the class
here at the University or you will not be reimbursed.}


\section{Schedule}

\begin{itemize}
\item Second Tuesday in the semester (Jan. 18):
  \begin{itemize}
    \item Heritage and design report and schedule for lab completions due.
      \begin{itemize}
        \item Preliminary thoughts on FTU, main science payload, and
          tracker should be included in the heritage design.
      \end{itemize}
    \item Team names due.
    \item Team leadership established.  Each lab should have someone
      associated with it (a person in charge). This person will then
      be responsible for turning the lab in on time, and I will bug
      this person about it. Everyone in the team should have a minimum
      of 1 lab to be in charge of, and at least 2 people should be
      assigned to each task.
  \end{itemize}

\item Thursday before Spring Break (Feb 24th):
  \begin{itemize}
    \item Everything should be working on breadboards by this point.
    \item You really should have labs 1-5 done before you leave on
      spring break. If you do not, then you are far behind schedule.
    \item PCB files must be turned in by this time.
  \end{itemize}

\item First Friday in April (April 1st):
  \begin{itemize}
    \item Flight Readiness Review must be completed by this date.
    \item Payload needs to be ready to go by the last First Friday in
      April or you will lose 1\% for each day beyond this needed to
      finish payload.
  \end{itemize}

\item Launch window is second week of April (2nd - 10th):
  \begin{itemize}
    \item If you launch later than April 10th, then you lose 1\% every
      day after this that you do not launch.
    \item As a reminder, all of the labs (except final report) MUST be
      completed before launching. There will be no exceptions.
    \item Typically, the weather in Michigan is not very cooperative,
      so it is crucial to have your system ready to fly by the start
      of the launch window. This ensures that if a good weather day
      arrives in the first few days of the launch window, your team
      can take advantage of it. I am not responsible for weather, so
      the earlier the launch is completed, the better. In planning,
      you should ALWAYS assume that you will be launching on the first
      Saturday in April. This will give you a bit of margin, but that
      margin is very costly. Remember this.
  \end{itemize}
        
\item Day of the Final Exam (Wed. April 27th 8 AM):
  \begin{itemize}
    \item Final reports (Lab 10) must be turned in before this time.
    \item All equipment must be checked back in (in working condition).
    \item Lab spaces must be cleaned.
  \end{itemize}
\end{itemize}

\section{Grading}

The class will be graded using the following criteria:

\begin{itemize}
  \item Heritage Report and Preliminary Design (10\%): This does not
    have to be very formal, but should outline what types of systems
    have been used before, why you are choosing one system over
    another, and a preliminary design for each of the three payloads.
  \item Documentation (55\%): This includes the labs discussed above,
    except Lab 9. It also includes the final report. Labs are worth
    different percentages, but the total is 55\%.
    \item System Testing - Lab 9 (10\%): While this could fall under
      documentation, it isn’t going to. There are a set of tests that
      must be completed before I will sign off on the balloon being
      launched. The proof of these tests will be turned in for a
      grade.
    \item Launch (10\%): While a successful launch is highly desired,
      it is not completely expected. What I look for on launch day is
      good planning and preparation - the team should be ready to go
      when they say they are going to go. The tests should all be
      completed. A good document that has the launch location, landing
      location and to do checklist should be done. All aspects of the
      launch day should be handled smoothly, with minimal prompting
      from me. No teams will be allowed to launch until I have
      verified all tests. It will take me at least 1 day to verify
      this. Plan accordingly.
    \item Team Feedback (5\%): During the semester, there will be
      plenty of opportunity for discussion with me on the team
      members. I expect to have no real surprises on the team feedback
      at the end of the semester.  Having said that, 5\% of your grade
      will be based on peer evaluations.
    \item Ham Radio Certification / Soldering Training (5\%): All
      students must be HAM radio certified or have taken the soldering
      challenge. One or the other must be completed by the end of
      February.
    \item Lab Stewardship (5\%): All teams must treat the lab space
      with the upmost respect. It is expected that the teams keep
      their areas organized, respect other team’s (and other lab!)
      equipment, and clean up their area when they are done with the
      semester.
    \item Penalties/Bonuses ($\pm 1$\% per day): If your payload is
      not ready by the first Friday of April and/or if you launch
      after the launch window closes, you will lose 1\% of your grade
      per day. If you launch before the first Friday in April, you
      will get 1\% bonus every day. If you go above 100\%, then we
      will come up with some other sort of compensation.
    \item We are pairing with Engin 100 teams, they will be providing
      payloads less then 1 lbs in weight, and you will have up to 4 of
      them on your payload train. You need to be well prepared to
      guide them through launch and in connecting their payload safely
      into the payload train.
\end{itemize}


\end{document}
